\documentclass[10pt]{beamer}

\usetheme[progressbar=frametitle]{metropolis}
\usepackage{appendixnumberbeamer}

\usepackage{booktabs}
\usepackage[scale=2]{ccicons}

\usepackage{bm}

\usepackage{pgfplots}
\usepgfplotslibrary{dateplot}

\usepackage{xspace}
\newcommand{\themename}{\textbf{\textsc{metropolis}}\xspace}

\title{Literature Research}
\subtitle{Simulation study of the formation of a non-relativistic pair shock (\emph{M. E. Dieckmann}, \textbf{J. Plasma Phys. 2017})}
\date{\today}
% \date{}
\author{Weipeng Yao}
\institute{Center for Applied Physics and Technology}
% \titlegraphic{\hfill\includegraphics[height=1.5cm]{logo.pdf}}

\begin{document}

\maketitle

\begin{frame}{Table of contents}
  \setbeamertemplate{section in toc}[sections numbered]
  \tableofcontents[hideallsubsections]
\end{frame}

\section{Introduction}

\begin{frame}[fragile]{Opening: Astrophysical jet}
  
  \begin{itemize}
  
  \item relativistic jet from microquasars -- observed directly
  
  \item ultra-relativistic jet from GRBs during supernovae -- not possible
  
%   \item mildly relativistic plasma outflow -- observed by Kulkarni et al. from supernova 1998bw
  
  \end{itemize} 

\end{frame}

\begin{frame}[fragile]{Physics: Internal shocks}
  
  \begin{itemize}
  \item WHY:
  
  acceleration efficiency due to accretion is not constant in time
  
  \item WHERE:
  
  locations with a large velocity change
  
  \item SO:
  
  strong sources of electromagnetic radiation (Rees 1978)
  
  \end{itemize}
  
\end{frame}

\begin{frame}[fragile]{Internal shock in GRBs jet}

  \begin{itemize}
  \item relativistic factors -- order of a few
  \item scenario -- collision of lepton clouds at relativistic speeds
  \item dominant -- magnetic fields from beam-Weibel/filamentation instability mediate these shocks and thermalize plasma
  \item \alert{a wide range of theoretical and numerical studies} -- 
  
  Kazimura et al. 1998; Medvedev \& Loeb 1999; Brainerd 2000; Sakai et al. 2000; Haruki \& Sakai 2003; Silva et al. 2003; Jaroschek, Lesch \& Treumann 2004; Medvedev et al. 2005; Milosavljevic \& Nakar 2006; Chang, Spitkovsky \& Arons 2008; Stockem, Dieckmann \& Schlickeiser 2008; Bret et al. 2008, 2013; Sironi \& Giannios 2014; Marcowith et al. 2016
  \item in other word, EM $>$ ES
  \end{itemize}

\end{frame}

\begin{frame}[fragile]{Internal shock in microquasar jet}
  \begin{itemize}
  \item component -- a significant fraction of positrons (Trigo et al. 2013)
  \item may not always be relativistic -- electrostatic processes may become important
  \item \alert{Motivation -- Non-relativistic pair shocks have so far not reveived much attention and the structure of their transition layers remains unknown.}
  \end{itemize}
  \textcolor{red}{Non-relativistic electron-ion shocks exist in solar winds studies.}
\end{frame}

\section{Shock formation, simulation setup and theoretical solution}

\subsection{formation mechanism of collisionless leptonic shock}

\begin{frame}[fragile]{Three wave modes}
  
  assumption -- charge and current neutral, pair components, unmagnetized, same temperature
  \begin{itemize}
  \item two-stream modes -- purely electrostatic (parallel)
  \item oblique modes -- quasi-electrostatic (oblique)
  \item filamentation/beam-Weibel modes -- electromagnetic (perpendicular)
  \end{itemize}
  grow simultaneously during the shock formation stage 
  
%   halted by nonlinear processes -- heating towards a thermal equilibrium
\end{frame}

% \begin{frame}{Three scenarios}
%   \begin{itemize}
%   \item reflecting wall -- efficient \\
%   shock formation phase may not be correctly resolved \\
%   suppress the filamentation instability at the reflecting wall
%   \item identical colliding clouds -- resolve formation stage \\
%   computationally expensive and unnecessary to track  both shocks for a long time \\
%   \item long \& short colliding clouds -- all good
%   \end{itemize}
% \end{frame}

\subsection{Simulation setup}

\begin{frame}{Model}
  
  \begin{center}
    $N_x = 1.9\times 10^4, N_y = 760, ppc = 25$ \\
    $L_x = 60 \lambda_e, L_y = 2.4 \lambda_e, t_{sim} = 120 \omega_p^{-1}$
  \end{center}
  
  \begin{figure}
    \includegraphics[]{model.pdf}
  \end{figure}
  
%   \begin{quote}
%     The simulation is stopped well before the end of the inflowing lepton cloud encounters the shock or before the leptons that are reflected by the boundary at $x=0.35L_x$ return to the shock.
%   \end{quote}

\end{frame}

\subsection{solution of linear dispersion relation}

\begin{frame}{theoretical purposes}
  \begin{enumerate}
    \item verify the box is large enough to resolve the competing unstable modes
    \item which one is the fastest for the selected initial conditions
    \item spectrum of the growing waves
  \end{enumerate}
\end{frame}

\setbeamertemplate{frame footer}{\textcolor{red}{Bret, Gremillet \& Dieckmann 2010}}
\begin{frame}{dispersion relation}

  \begin{itemize}
    \item assumptions -- the overlap layer has an infinite size
    \item cold beam approximation -- $T_0 = 10eV, v_{th} = 4.5e-3c \ll v_0=0.2c$
    \item change of reference frame -- total momentum vanishes ($v_{l0} = v_{r0} = 0.1c$)
  \end{itemize}
  
  \begin{equation*}
    (\omega^2\epsilon_{xx}-k_y^2c^2)(\omega^2\epsilon_{yy}-k_x^2c^2) - (\omega^2\epsilon_{yx}+k_xk_yc^2)^2 = 0
  \end{equation*}
  where $\delta_{\alpha\beta}$ is the Kronecker symbol and 
  
  \begin{equation*}
    \epsilon_{\alpha\beta}(\bm{k},\omega) = \delta_{\alpha\beta}\left( 1-\frac{\omega_p^2}{\omega^2} \right) + \frac{\omega_p^2}{\omega^2} \sum_j \int d^3p \frac{p_{\alpha}p_{\beta}\bm{k}\cdot \left( \frac{\partial f_j^0}{\partial \bm{p}} \right)}{m\omega-\bm{k}\cdot\bm{p}}
  \end{equation*}
\end{frame}


\begin{frame}{box size limitation}
  \begin{figure}
    \includegraphics[width=0.6\textwidth]{solution.png}
  \end{figure}
  \begin{itemize}
  \item growth rate peaks at $k_x\lambda_e \approx 12$ and does not depend on $k_y\lambda_e$
  \item the smallest resolved wavenumber is $k_c = 2\pi/L_y$ or $k_y\lambda_e=2.6$ -- waves with $k_y < k_c$ cannot grow.
  \item for $k_y < k_c$, the growth rate of filamentation mode decreases below $\delta_{W}$, while that of TS/O remains unchanged.
  \item So, limited box size effect is to suppress the wavenumbers where the growth rate of F is negligible.
  \end{itemize}
\end{frame}

\setbeamertemplate{frame footer}{ }
\begin{frame}{solutions}

  \begin{quote}
    The problem of finding the fastest-growing mode has been solved (\textcolor{red}{Bret \& Deutsch 2005; Bret et al. 2013}) for cold distributions of the form $f_j^0(\bm{p}) = \delta(p_y)\delta(p_x-P_j)$.
  \end{quote}
  
  Another assumption -- identical density between two colliding electrons
  
  \begin{equation*}
      \begin{aligned}
        \varepsilon_{xx} & = 1-\frac{\omega_p^2}{\omega^2} + \frac{\omega_p^2}{\omega^2}\sum_j \frac{k_x^2p_0^2-k_y^2p_0^2-2k_xp_0m\omega}{(m\omega-k_xp_0)^2} \\
        \varepsilon_{yy} & = 1-\frac{\omega_p^2}{\omega^2} \\
        \varepsilon_{yx} & = -\frac{\omega_p^2}{\omega^2}\sum_j\frac{k_yp_0}{m\omega-k_xp_0}
      \end{aligned}
  \end{equation*}
  
  \textcolor{red}{To Do Numerical analysis to draw Fig.1}
  
  

\end{frame}

\begin{frame}{growth rate, separately}
    \begin{itemize}
      \item for two-stream/oblique modes, let $k_x \neq 0, k_y = 0$ \\
      $\frac{\delta_{TS}}{\omega_p}=\frac{\sqrt{2}}{2}$ \\
      \textcolor{red}{there's something wrong}
      \item for filamentation modes (flow-aligned $k_x=0, k_y \neq 0$) \\ $\frac{\delta_{W}}{\omega_p}=\textcolor{red}{2}\beta_0^{'}=0.2$
      \item $\delta_{TS} > \delta_{W}$, TS dominates.
      
      \item however, the fastest mode is oblique inst. when $k_x, k_y \neq 0$
  \end{itemize}
\end{frame}

\begin{frame}{Bret \& Deutsch 2005 -- 1}
the general form of the dispersion equation
\begin{equation}
    det \left| \frac{\omega^2}{c^2}\varepsilon_{\alpha \beta} + k_\alpha k_\beta - k^2\delta_{\alpha \beta}\right| = 0
\end{equation}
where the dielectric tensor elements are
\begin{equation}
    \varepsilon_{\alpha \beta}(\bm{k},\omega) = \delta_{\alpha\beta} + \frac{\omega_p^2}{n_e\omega^2} \int \frac{p_\alpha}{\gamma}\frac{\partial f_0}{\partial p_\beta}d^3p + \frac{\omega_p^2}{n_e\omega^2} \int \frac{p_\alpha p_\beta}{\gamma} \frac{\bm{k}\cdot \partial f_0 / \partial \bm{p}}{m\gamma\omega-\bm{k}\cdot\bm{p}}d^3p
\end{equation}

In the non-relativistic collisionless limit, the first quadrature reduces to $-\delta_{\alpha \beta}\omega_p^2/\omega^2$.  (\textcolor{red}{?})
\end{frame}

\begin{frame}{Bret \& Deutsch 2005 -- 2}
As long as the equilibrium function can be expressed as $\sum_i g_x^i(v_x^2) g_y^i(v_y^2) g_z^i(v_z)$, the electromagnetic dispersion equation for the TSF branch reads
\begin{equation}
    (\eta^2\varepsilon_{xx}-k_z^2)(\eta^2\varepsilon_{zz}-k_x^2)-(\eta^2\varepsilon_{xz}+k_xk_z)^2 = 0
\end{equation}
where $\eta = \omega/c$ and $\bm{k}=(k_x,k_z)$.

\begin{itemize}
\item Setting $k_x = 0$, TS
\item Setting $k_z = 0$, F
\item \Large $\delta_m^{TSF} \sim \omega_p\frac{\sqrt{3}}{2^{4/3}}\left(\frac{n_b/n_p}{\gamma_b}\right)^{1/3} $  
\end{itemize}
\end{frame}

\begin{frame}{Bret \& Deutsch 2005 -- 3}

\begin{equation}
\begin{aligned}
    f_0 & = f_0^p + f_0^b \\
    f_0^p & = n_p\delta(p_x)\delta(p_y)\delta(p_z+P_p) \\
    f_0^p & = n_b\delta(p_x)\delta(p_y)\delta(p_z-P_b)
\end{aligned}
\end{equation}
where $P_p = mV_p$ (return impulsion) and $P_b=m\gamma_bV_b$ (beam impulsion), assuming $n_pV_p = n_bV_b$.

\begin{equation}
\begin{aligned}
0 & = \left[ \frac{Z_xZ_z}{\beta^2} + \frac{\alpha Z_x}{x+\alpha Z_z} - \frac{\alpha Z_x}{\gamma_b(x-Z_z)}  \right]^2 \\
  & + \left[ -\frac{Z_z^2}{\beta^2} + x^2 - 1 - \frac{\alpha}{\gamma}  \right] \\
  & \times \left[ -\frac{Z_z^2}{\beta^2} + x^2 - \frac{x^2+\alpha^2Z_x^2}{(x+\alpha Z_z)^2} - \frac{\alpha}{\gamma_b^3} \frac{x^2+\gamma_b^2Z_x^2}{(x-Z_z)^2} \right]
\end{aligned}
\end{equation}
The following dimensionless variables are
\begin{equation}
\bm{Z} = \frac{\bm{k}V_b}{\omega_p}, \alpha=\frac{n_b}{n_p}, \beta = \frac{V_b}{c}, x = \frac{\omega}{\omega_p}
\end{equation}
\end{frame}

\begin{frame}{Bret \& Stockem \& Fiuza \& Silva, et al. POP (2013)}
  \begin{itemize}
  \item \textcolor{red}{oblique to filamentation transition}
  \item The first derivative $\partial \delta_{Z_\perp, \infty}/\partial Z_\parallel$ always vanishes for $Z_\parallel = 0$.
  \item The transition from one regime to the other occurs then when the second derivative vanishes at $Z_\parallel = 0$.
  \item After some derivations -- $\gamma_0 = \sqrt{3/2}$, $\beta_0 = 1/\sqrt{3}$.
  \end{itemize}
\end{frame}




\section{Simulation results}

\subsection{Instability and nonlinear saturation}

\begin{frame}{ES $>$ EM}

\begin{figure}
  \includegraphics[width=0.6\textwidth]{fig2.png}
\end{figure}
\begin{center}
\begin{itemize}
\item $v_0t_1=1.5$
\item from $E_x$, the wavelength $\lambda \approx 0.4$, and do not propagate along $y$.
\item $E_y$ closely correlated to $B_z$, and keep $|E_y|/|B_z| \sim 5$ -- \textbf{electric force much larger than magnetic force}
\end{itemize}
\end{center}
\end{frame}

\begin{frame}{my fig2}

\begin{figure}
  \includegraphics[width=0.3\textwidth]{f2d.png}
  \includegraphics[width=0.3\textwidth]{f2e.png}
  \includegraphics[width=0.3\textwidth]{f2f.png}
\end{figure}

\end{frame}


\begin{frame}{in line with theory}
  \begin{itemize}
  \item from $t_2-t_1=7.1$, wave amplitude increased by $50$ -- growth rate $\delta \approx 0.5$
  \item wavelength $\lambda \approx 0.4$ -- $k_x = 2\pi/\lambda = 15$
  \item \textcolor{red}{spatial power spectrum}
  \end{itemize}
  
  \begin{figure}
  	\includegraphics[width=0.5\textwidth]{fig3.png}
  \end{figure}
  
  \begin{quote}
    the range of wavenumbers $k_y$ that are unstable to the oblique mode instability is large in a cold plasma, while the wave growth is concentrated at low values of $k_y$ if the plasma is hot.
  \end{quote}
\end{frame}

\begin{frame}[fragile]{my fig3}
\begin{figure}
  \includegraphics[width=0.45\textwidth]{f3a.png}
  \includegraphics[width=0.45\textwidth]{f3b.png}
\end{figure}

% Ranges of $k_x$ \& $k_y$ are determined by $dx$ \& $dy$, for 
% \begin{verbatim} 
%   kx = ((0:nx-1)-nx/2)/(nx*dx) 
% \end{verbatim}
% so $k_{x,max} = \frac{1}{2d_x}$, which is \textcolor{red}{158} in the paper's parameters.

% Here, my $k_x$ is much smaller than his, and I don't know why.
\end{frame}

\begin{frame}{nonlinear effects}
  \begin{itemize}
  \item harmonic along $k_x$ have emerged.
  \item lepton phase space distribution $f(x,p_x)$.
  \end{itemize}

  \begin{figure}
    \includegraphics[width=0.7\textwidth]{fig4.png}
  \end{figure}
\end{frame}

\begin{frame}{my fig4}
\begin{figure}
    \includegraphics[width=0.45\textwidth]{f4a.png}
    \includegraphics[width=0.45\textwidth]{f4c.png}
  \end{figure}
  
  seems particle number is not enough, comparing to Dieckmann's.
\end{frame}

\subsection{Shock formation}


\begin{frame}{Saturation: phase space vortices}
  \begin{itemize}
  \item $2<x<4$: strong bipolar pulses in electric field dist.
  \item $x>4$: thermalized downstream region
  \item vortices of positrons and electrons are staggered along x.
  \item $x=0$ \& $p_x=0$: initial right component -- instability starts after inter-penetration.
  \end{itemize}
  \begin{figure}
    \includegraphics[width=0.7\textwidth]{fig5.png}
  \end{figure}
\end{frame}

\begin{frame}{my fig5}
  \begin{figure}
    \includegraphics[width=0.35\textwidth]{f5a.png}
    \includegraphics[width=0.35\textwidth]{f5b.png} \\
    \includegraphics[width=0.35\textwidth]{f5c.png}
    \includegraphics[width=0.35\textwidth]{f5d.png} \\
  \end{figure}
  
\begin{center}  
\begin{itemize}
  	\item upstream inflow: fast, cold
    \item downstream: heated, mixed
    \item transition layer: downstream vortex \& upstream oscillation
    \item penetrating: slow, hot
  \end{itemize}
\end{center}
\end{frame}

\begin{frame}{Dieckmann's fig6}

\begin{figure}
\includegraphics[width=0.6\textwidth]{fig6.png}
\end{figure}


\begin{center}  
\begin{itemize}
%   	\item \footnotesize $x<-8$ \& $x>8$: stable
    % \item \footnotesize $-6 < x < -5$: non-thermal -- \textcolor{red}{todo spectrum}
    \item \footnotesize $-3 < x < 0$:
    \begin{itemize}
    \item \footnotesize penetrating component from hot/slow downstream
    \item \footnotesize modulated upstream -- \textcolor{blue}{ponderomotive force} -- accelerate vortex
    \end{itemize}
    \item \footnotesize $0 < x < 6$: mixed up/downstream, and merge at $x=6$ -- piecewise planar struct. that lead to vortex
    \begin{itemize}
    \item \footnotesize the collapse of a ps vortex is an efficient way to scatter the leptons in $x,p_x$ space. \textcolor{red}{(Oppenheim, Newman \& Goldman, 1999)}
    \end{itemize}
  \end{itemize}
\end{center}
\end{frame}


\begin{frame}{my fig6}
\begin{figure}
\includegraphics[width=0.35\textwidth]{f6a.png}
\includegraphics[width=0.35\textwidth]{f6b.png} \\
\includegraphics[width=0.35\textwidth]{f6c.png}
\includegraphics[width=0.35\textwidth]{f6d.png} \\
\end{figure}
\end{frame}

\begin{frame}{Dieckmann's fig7}

\begin{figure}
\includegraphics[width=0.6\textwidth]{fig7.png}
\end{figure}

\begin{itemize}
\item $x<0$: converges to initial density
\item $x\approx 0$: density peak in (a) and disappear in (b)
\begin{itemize}
\item from fig.5(c), undisturbed rightside component
\item from fig.6(a), large $E_x$ tore it apart by accelerating $e^{\pm}$ into opposite direction
\end{itemize}
\item $x>0$: compression ratio is 3 -- strong non-relativistic shock \textcolor{red}{(Zel'Dovich, 1967)}
\end{itemize}
\end{frame}

\begin{frame}{my fig7}
\begin{figure}
\includegraphics[width=0.35\textwidth]{f7a.png}
\includegraphics[width=0.35\textwidth]{f7b.png} \\
\includegraphics[width=0.35\textwidth]{f7c.png}
\includegraphics[width=0.35\textwidth]{f7d.png} \\
\end{figure}

\begin{itemize}
\item from c\&d: the spatial-temporal scale is designed dedicatedly.
\item from b: \textcolor{blue}{density oscillation represents field oscillation, which means ponderomotive force and phase space vortex, leading to plasma heating.}
\end{itemize}
\end{frame}


\subsection{Secondary instabilities and magnetic field generation}



\begin{frame}{lepton heating}
\begin{figure}
\includegraphics[width=0.6\textwidth]{fig8.png}
\end{figure}

\begin{itemize}
\item far upstream: cold dense beam ($p_0$) \& hot dilute beam ($-p_0$)
\item transition layer: heated inflow \textcolor{red}{($\sim 100$ eV)} \& hot dilute right component \textcolor{red}{($\sim 1$ keV)} -- \textcolor{blue}{How?}
\item downstream: hot beam
\item \textcolor{red}{Integrated over y}?
\end{itemize}
\end{frame}

\begin{frame}{More analysis}
\begin{itemize}
\item transition layer: unstable from $E_x$ and $E_y$.
\begin{itemize}
\item not TS -- require two colliding flows well separated in phase space.
\item \textcolor{red}{electron acoustic instability (Gary 1987)} -- not yet been explored in pair plasma
\end{itemize}
\item downstream: stable because no $E_x$ and $E_y$.
\begin{itemize}
\item velocity dist. is not Maxwellian.
\item free energy that can be released by collisionless instability -- \textcolor{red}{TODO energy evolution}
\end{itemize}
\item anisotropy: $A = \frac{\int f_t(v_x,v_y)(v_x-p_0/2m_e)^2dv_xdv_y}{\int f_t(v_x,v_y)v_y^2dv_xdv_y} -1 \approx 6 > 0$ -- Weibel instability
\end{itemize}
\end{frame}

\begin{frame}{my fig8}
\includegraphics[width=0.3\textwidth]{f8a.png}
\includegraphics[width=0.3\textwidth]{f8b.png} 
\includegraphics[width=0.3\textwidth]{f8c.png} \\
\includegraphics[width=0.3\textwidth]{f8d.png} 
\includegraphics[width=0.3\textwidth]{f8e.png}
\includegraphics[width=0.3\textwidth]{f8f.png} \\

\begin{itemize}
\item Need more particles in my simulation. ($random\_factor = 0.01$)
\item \textcolor{red}{What is that slash???}
\end{itemize}
\end{frame}

\begin{frame}{downstream magnetic field}
\begin{figure}
\includegraphics[width=0.4\textwidth]{fig9.png}
\end{figure}

\begin{itemize}
\item \footnotesize strong magnetic field: 
\begin{itemize}
\item \footnotesize amplitudes $\sim E_y$
\item \footnotesize not correlated with $E_y$ -- not by oblique mode
\item \footnotesize by WBI:  (\textcolor{red}{?})
\begin{itemize}
\item \footnotesize generate magnetowaves with a negligible $E$
\item \footnotesize $\varepsilon_B/\varepsilon_k = 10\%$ -- $T_e$ from fig.8c and $B_z$ from fig.9b
\end{itemize}
\end{itemize}
\item \footnotesize $r_g/\lambda_e = v_0/2B_z \approx 3$ and $T_g = 2\pi/B_z \approx 200$:
\begin{itemize}
\item \footnotesize leptons can not complete a full gyro-orbit
\item \footnotesize instead, they are deflected by a small angle, thus being heated, as shown in fig.8c
\end{itemize}
\end{itemize}
\end{frame}

\begin{frame}{my fig9}
\includegraphics[width=0.3\textwidth]{f9a.png}
\includegraphics[width=0.3\textwidth]{f9c.png} 
\includegraphics[width=0.3\textwidth]{f9e2.png} \\
\includegraphics[width=0.3\textwidth]{f9b.png} 
\includegraphics[width=0.3\textwidth]{f9d.png}
\includegraphics[width=0.3\textwidth]{f9f2.png} \\
\end{frame}

\section{Conclusion}

\begin{frame}{Summary -- done}

\begin{enumerate}
\item shock: initial evolution ($120/\omega_p$) \& non-relativistic ($0.2c$) \& equally dense \& pair
\item phase space vortex (escaping downstream with up. inflow): unstable and heating by \textcolor{red}{collapse}
\item NOT fully thermalize the up. that crossed it -- anisotropy -- WBI
\begin{itemize}
\item magnetic field patches will grow until lepton reach thermal equilibrium
\item \textcolor{red}{which may be the saturation mechanism of $\varepsilon_B$ in energy evolution}
\end{itemize}
\end{enumerate}
\end{frame}

\begin{frame}{Summary -- to be done}
\begin{enumerate}
\item for internal shock of microquasar jet: larger $v$ and $T$ -- up-limit of ES-shock
\item detail spectrum of the unstable waves in transition layer ($p_x-p_y$) -- electron acoustic instability
\end{enumerate}

\end{frame}



%{\setbeamercolor{palette primary}{fg=black, bg=yellow}
\begin{frame}[standout]
  Questions?
\end{frame}
%}

%\appendix

% \begin{frame}[fragile]{Backup slides}
%   Sometimes, it is useful to add slides at the end of your presentation to
%   refer to during audience questions.

%   The best way to do this is to include the \verb|appendixnumberbeamer|
%   package in your preamble and call \verb|\appendix| before your backup slides.

%   \themename will automatically turn off slide numbering and progress bars for
%   slides in the appendix.
% \end{frame}

% \begin{frame}[allowframebreaks]{References}

%   \bibliography{demo}
%   \bibliographystyle{abbrv}

% \end{frame}

\end{document}
